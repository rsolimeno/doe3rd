% Fractional Factorial Designs

\chapter{Fractional Factorial Designs}
In many situations the key process parameters are largely unknown, and yet a technologist may presume that there are five, eight, ten, or more factors that are candidates.  As we saw in the last chapter, the size of factorial experiments grows exponentially by design. The two level designs grew from 4 to 8, then from 8 to 16 runs \textit{without replication} just by increasing the number of candidate factors from two to three, and then to four factors.  It would seem nearly impossible to evaluate a larger number of factors and it would presumably take an enormous amount of time and resources to evaluate all of the factors in a full factorial design.

Fractional Factorial Designs address this problem directly.  Nothing comes without a price, however, and the fractional designs that are discussed in this chapter are no exception. As the name implies, only a \textit{fraction} of the full factorial matrix is actually evaluated with the commensurate savings in terms of the number of experimental runs and associated testing to obtain response data.  In exchange for this convenience, some of the information that is gleaned from a full factorial design is sacrificed.  When done with care these designs can yield very coherent results about main effects, and sometimes two-factor interactions, from a large filed of factors.  In essence, fractional designs are very effective \textit{screening designs}.

\section{Creating Fractional Designs}
When experimental situations present four, five, or more factors to be studied the possibility of using a fractional factorial should always be considered. A fractional factorial design is comprised of a sub-set of the runs in a full factorial design for a large number of factors.  These designs allow investigation of a larger number of factors in a relatively small number of runs.  As the number of runs is reduced, however there is a loss of ability to independently estimate some effects.  In fractional designs some of the effects are ''\textit{aliased}'' meaning that they are indistinguishable from each other, or confounded. Consequently, when developing a fractional design it is important to determine and understand the nature of the aliases that will exist between certain effect estimates prior to carrying out the experiment.

Fractional factorial designs are meant to be efficient screening experiments.  The purpose of such experiments is to study a relatively large number of candidate factors with a goal of identifying the few that are really key in affecting the process under study. This advantage of factorial designs becomes more valuable as the number of factors to study increases.
\begin{table}[h]\caption{The $2^{4-1}$ factorial design}\label{tab9}
\begin{center}
\begin{tabular}{|l|c|c|c|c|}
\hline Run & A & B & C  & D  \\ 
\hline 1 & -- & -- & -- & -- \\ 
\hline 2 & -- & -- & +  & +  \\ 
\hline 3 & -- & +  & -- & +  \\ 
\hline 4 & -- & +  & +  & -- \\
\hline 5 & +  & -- & -- & +  \\
\hline 6 & +  & -- & +  & -- \\
\hline 7 & +  & +  & -- & -- \\
\hline 8 & +  & +  & +  & +  \\ 
\hline 
\end{tabular} 
\end{center}
\end{table}

\subsection{Creating a $2^{4-1}$ fractional factorial design}\label{createfrac}
To create a one-half fraction of a $2^{4}$ full factorial (16 runs, see section \ref{4fac}) only half of the full matrix would be run in 8 runs.  A factional design like this is denoted as a $2^{4-1}$ design - and if you do the math the exponent reduces to 3 and thus the exponential term is just like a 3-factor full factorial design -- with 8 runs.  The difference here is that there are actually \textit{four factors} being evaluated instead of just three in the same number of runs. To generate the $2^{4-1}$ fractional design the first three factors A, B, and C are used to establish the base design, and the column for the fourth factor D is generated by taking the product of the columns A, B, and C. Table \ref{tab9} provides a view of the fractional design matrix. The \textbf{design generator} is
\begin{equation}\label{desgen}
D = ABC
\end{equation}

An important property of these designs that will come into play is that any factor (column) times itself (squared) is equal to a column of +'s which is called the identity column, I. An example of this with the newly generated column D will serve to demonstrate:

\begin{center}
\begin{tabular}{ccccc}
 D  & x & D  & =  & I \\ 
 -- & x & -- & =  & + \\ 
 +  & x & +  & =  & + \\ 
 +  & x & +  & =  & + \\ 
 -- & x & -- & =  & + \\
 +  & x & +  & =  & + \\
 -- & x & -- & =  & + \\
 -- & x & -- & =  & + \\
 +  & x & +  & =  & + \\ 
 
\end{tabular} 
\end{center}


The design generator given in equation \ref{desgen}  is now used by multiplying each side by factor D to yield the \textbf{defining relation} given in equation \ref{defreln}
\begin{equation}\label{defreln}
D^{2} = ABCD = I
\end{equation}

The defining relation is used  to uncover aliases in the design. To find all aliases, just multiply both sides of the defining relation by the factor of interest. For example, to find the aliases of factor A multiply both sides of the defining relation in equation \ref{defreln} by A as shown in equation \ref{aliasA} below:
\begin{equation}\label{aliasA}
A \cdot I = A \cdot ABCD
\end{equation}

Anything multiplied by the identity, I, is like multiplying a value by 1 so that it returns the same value.  Thus the left side of equation \ref{aliasA} is  $A \cdot I = A$.  The right side of the equation becomes $A^{2}BCD = BCD$ since we already know that $A^{2} = I$.  Thus $A = BCD$ which means that the alias for A in this design is the ternary interaction BCD.  Since we have already learned about the sparsity of effects principle (section \ref{sparsity}) we know that BCD is a very rare event and so this alias is ''ok'' -- any BCD effect aliased with A will be insignificant.  So if the results indicate that A yields a strong effect, we can believe that the effect is really due to A and not BCD.

\subsection{Design classification}
Factorial designs are classified by their resolution.  The resolution of a fractional factorial design indicates the nature of the aliasing structure in the design. Resolution III designs are those in which main effects are not aliased with each other but main effects are aliased with two-factor interactions. Such designs may be useful for screening large numbers of variables for a quick screening of potentially large main effects. Care must be taken here since any aliased factors cannot be resolved without additional runs carried out.

Resolution IV designs are those in which the main effects area aliased with three-factor interactions, and two-factor interactions are aliased with each other.  These designs are very useful in screening situations where there is interest in studying the main effects and to get a sense of the degree of two-factor interaction effects.  Once again be aware of which terms are aliased to guide interpretation of results. The $2^{4-1}$ fractional design in section \ref{createfrac} is an example of a resolution IV design. An alternative notation that indicates resolution for this design is $(2^{4-1}_{IV})$.

Resolution V fractional factorial designs are those in which main effects are aliased with four-factor interactions, and two-factor interactions are aliased with three factor interactions.  The main effects and two-factor interactions are the main focus of these designs. Any design that is resolution V can be used without concern for the aliased terms since they are restricted to the high order interactions. The general notation for such a design is $(2^{n-k}_{V})$ and an example design is a five-factor factorial in 16 runs, or a $(2^{5-1}_{V})$ design.  This is an excellent design to use and whenever an experimental situation offers at least four factors to study it would be worthwhile to look for a fifth - the fifth factor can be added on for ''free.''  There are still 16 runs just as with the four-factor full factorial design, but five factors can be evaluated with main and two-factor effects aliased with sparse interaction effects (four-factor and three-factor, respectively) that can be neglected.

\section{EXAMPLE: Chemical Manufacturing -- a $2^{5-1}_{V}$ factorial design}
A chemical producer is interested in maximizing the yield of product manufactured at one of their chemical plants. The process engineers settled on the following factors thought to have an influence on the yield of the product:
\begin{enumerate}
\item the temperature,
\item the agitation rate of the reactor,
\item the percentage of the catalyst, 
\item the feed rate of the chemicals, and
\item the concentration.
\end{enumerate}

A $(2^{5-1}_{V})$ design could give them all of the relevant information that they could get from a $2^{5}$ full factorial. The fractional design could be accomplished with half of the number of runs (16 vs. 32).  This more efficient, lest costly experiment design became the clear choice.

To generate the design, the process engineer used the design generator $E = ABCD$ to get the design matrix shown in Table \ref{tab10}.  The factors with their respective high and low levels are given in Table \ref{tab11}.

\begin{table}[h]\caption{The $2^{5-1}_{V}$ factorial design}\label{tab10}
\begin{center}
\begin{tabular}{|l|c|c|c|c|c|}
\hline Run& A  & B  & C  & D  & E  \\ 
\hline 1  & -- & -- & -- & -- & +  \\ 
\hline 2  & -- & -- & -- & +  & -- \\ 
\hline 3  & -- & -- & +  & -- & -- \\ 
\hline 4  & -- & -- & +  & +  & +  \\
\hline 5  & -- & +  & -- & -- & -- \\
\hline 6  & -- & +  & -- & +  & +  \\
\hline 7  & -- & +  & +  & -- & +  \\
\hline 8  & -- & +  & +  & +  & -- \\ 
\hline 9  & +  & -- & -- & -- & -- \\ 
\hline 10 & +  & -- & -- & +  & +  \\ 
\hline 11 & +  & -- & +  & -- & +  \\ 
\hline 12 & +  & -- & +  & +  & -- \\
\hline 13 & +  & +  & -- & -- & +  \\
\hline 14 & +  & +  & -- & +  & -- \\
\hline 15 & +  & +  & +  & -- & -- \\
\hline 16 & +  & +  & +  & +  & +  \\ 
\hline 
\end{tabular} 
\end{center}
\end{table}

\begin{table}[h]\caption{Chemical Manufacturing Experiment Design}\label{tab11}
\begin{center}
\begin{tabular}{|l|c|c|c|c|}
\hline Run & Low Level (--) & High Level (+)  \\ 
\hline A - Temperature ($^{o}C $) & 140 & 180  \\
\hline B - Agitation rate (rpm) & 100 & 120 \\
\hline C - Percentage of catalyst (\%) & 1 & 2 \\
\hline D - Feed rate (liters/min) & 10 & 15 \\
\hline E - Concentration (\%) & 3 & 6 \\
\hline
\end{tabular} 
\end{center}
\end{table}

\subsection{Analysis of the response data}
Since the significance of the higher order interaction effects are very sparse, only the main effects and two-factor interaction effects were analyzed (see Figure \ref{fig10}). Based on the magnitude of calculated effects, main effects A, C, and (to a lesser extent) E stand out, and interaction effects AC and DE are significant contributors. The significance of AC is not surprising given that both A and C main effects are also strong. It is interesting that DE is quite strong since main effect D is not significant, and the magnitude of main effect E is less than the effect of DE.  This suggests some synergy may be occurring.

\begin{sidewaysfigure}[h]\caption{Chemical Manufacturing Experiment Analysis Matrix}\label{fig10}
\begin{center}
\includegraphics[width=\textwidth]{ChemMfg}
\end{center}
\end{sidewaysfigure}

In addition to simple inspection of the effect values, it is very helpful to use a graphic device for analysis.  Figure \ref{fig11} gives the \textbf{normal probability plot} of the same data with the effects sorted in ascending order.  This plot was generated in a spreadsheet.  Interpretation is aided by the addition of a trend line that fits the normally distributed points (mostly falling on a straight line in the center of the plot).  The outliers are those effects with a significant ''signal'' -- if any given effect is just random noise, then these (averaged values) are normally distributed.

\begin{figure}[h]\caption{Chemical Manufacturing Experiment Normal Probability Plot}\label{fig11}
\begin{center}
\includegraphics[height=3 in, width=3 in]{ChemMfg-norm}
\end{center}
\end{figure}

So as one can see in Figure \ref{fig11} the same effects discussed above namely, A, C, and interaction AC appear as ''outliers'' at the top right, and DE and E appear on the lower left.  In practice it is best to do both. Render a decision about effects that are relatively small (closer to the normal line) only after both effect magnitudes and the normal probability plot have been inspected.\\


A comparison of the interaction effects in Figure \ref{fig12} illustrates the previous point about relative magnitudes of effects versus proximity to the normal line.  Interaction AC is the larger effect and shows that when factors A and C are both set at their high levels there is a synergistic effect and the chemical yield approaches 90\% conversion.  In contrast the DE interaction is more mild, having a smaller slope and with factors D and E both set low there is only about a 3\% gain in yield over setting D to the high level (factor E adversely affects yield across the board when set high).

\begin{sidewaysfigure}[h]\caption{Chemical Manufacturing Experiment Interaction Plots}\label{fig12}
\begin{center}
\includegraphics[width=\textwidth]{ChemMfg-int}
\end{center}
\end{sidewaysfigure}



