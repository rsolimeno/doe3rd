% Intro to DOE
\chapter{Introduction to DOE}
This chapter is an introduction to experimental design, also known as ``design of experiments'' -- abbreviated \textit{DOE}. Students will learn the value of applying experimental design principles versus a one-factor-at-a-time approach. Full factorial and fractional factorial designs are covered with in-class experiments using this methodology.

Performing an experiment is basically a test to see how the process, instrument, or device \textit{responds} to deliberate changes in a number of settings, or \textit{factors}.  The investigator's interest is in observing how the deliberate changes of the input variables (factors) affect the output variables (responses).  In the analysis of the experimental data the investigator may identify reasons for changes in the responses such that a better understanding of the system under study is achieved.

Planning and conducting experiments in a structured way and analyzing the results of such a study provide for valid and objective conclusions to be made.  The strategies used in this type of experimentation reveal to what extent a variety of factors influence the responses of interest, and in addition also reveal if there are any \textit{interaction}  effects between the factors that are significant.

The alternative strategy of experimentation that is widely practiced is the one-factor-at-a-time approach. In this method a base set of levels for each factor is established, and then each factor is varied over its range while all other levels are held constant. Graphical analysis is typically used to compare the responses for each factor to each other when varied over the range of interest with all others being held constant. The major disadvantage of this approach is that it fails to consider any possible interaction between the factors.  Interactions between factors are very common!

The correct approach in dealing with multiple input settings is to conduct a \textbf{factorial experiment} in which \textit{the settings are varied together} instead of one at a time.  This experimental design concept is of crucial importance and the detailed examples and exercises that follow will serve to demonstrate how to carry out such a plan, how to analyze the data, and interpret the results.

