% Full Factorial Designs
\chapter{Full Factorial Designs}

\section{Learning by experimentation}
We do experiments to test a hypothesis -- something we think might be true but have no evidence to demonstrate the idea.  The process of experimentation is a \textit{learning} process.  The experiments that this course offers for study are those that are intended to improve the quality of product designs and production (or measurement) processes.

To ''improve the quality'' means:

\begin{itemize}
\item Optimizing the mean value of response(s) -- a traditional focus of experimental designs.  For example, maximize yield of a chemical reaction, optimizing the opacity of printing paper, or maximizing the distance a catapult launches a projectile. In each of these examples there are various settings involved in making the process work that may be adjusted to different levels or settings.  Adjustment of these parameters in relation to each other can achieve the desired end result -- and is the object of the experiment.

\item Minimizing variation in a process or product -- variation and quality of performance are inversely related. There is always some variation present in any process, i.e. random variation that is uncontrollable, and the role of experimentation in reducing variation is to identify and reduce those sources of variation that are not random and are controllable.
\end{itemize}

The definition of an experiment used here is a series of trials or tests that produce quantifiable outcomes. While the criterion for quantifiable outcomes is established, the initial data need not be numerical results. Pass/fail results, or categorical results established with Excellent/Good/Fair/Poor judgements can be converted to numerical results by assigning values to each category.

\section{Traditional methods}
Some scientists or engineers with many years of experience might say, "I know how to test products - just vary one thing at a time and keep everything else constant so you can see the effect of what you are changing."  While one variable at a time experiments have an intuitive appeal and seem quite straight-forward to plan and do, they are not good experiment designs.  These types of experiments generally:

\begin{enumerate}
\item lead to misleading conclusions,
\item cannot reveal any possible interaction effects,
\item may not result in determining the true optimum conditions, and
\item are usually inefficient.
\end{enumerate}

There is a better way.  Factorial experiments are those in which every level of one factor is run in combination with all levels of the other factors.  Thus all possible combinations are run in a factorial design experiment. It should be understood that the factors being studied in these types of experiments are \textit{independent variables}.  These are usually process settings such as temperature, line speed, tool size, material substitution, etc.  

For experiments with many factors this can lead to a large number of runs, or tests.  There are several ways of dealing with these situations to reduce the number of runs in the factorial experiment while preserving all of the useful information. These methods are dealt with in the next chapter.

\section{Two-factor design}
\begin{table}[h] \caption{The $ 2^{2} $ factorial design }\label{tab1}
\begin{center}
\begin{tabular}{|l|c|c|}
\hline Run & A & B \\ 
\hline 1 & -- & -- \\ 
\hline 2 & -- & \cellcolor{black!25}+ \\ 
\hline 3 & \cellcolor{black!25}+ & -- \\ 
\hline 4 & \cellcolor{black!25}+ & \cellcolor{black!25}+ \\ 
\hline 
\end{tabular} 
\end{center}
\end{table}
The most common full factorial designs are two-level experiment designs denoted as $ 2^{n} $ designs, where $2$ denotes the number of levels for each factor to be evaluated and $n$ denotes the total number of factors under study.  The simplest of these is the 2-factor, 2-level design, or the $ 2^{2} $ design that involves 4 sets of conditions, or runs, to complete the experiment.  This number of experimental runs is determined by computing the exponential expression $ 2^{2} = 4 $.

In Table \ref{tab1}, the ''--'' denotes a low level of the corresponding factor (A or B) and the ''+'' denotes a high level of the factor.  Hence the name ''two-factor, two-level design'' where A and B are the two factors and ''--'' and ''+'' denote the two levels. In the example that follows we will run an experiment, collect individual observations of repeated runs, and analyze the data ``manually'' (and is easily done with spreadsheet software).

\subsection{EXAMPLE:The Pump Experiment} 
\label{router}
A diaphragm pump is one in which a diaphragm, made of or coated with a chemically resistant material (such as PTFE), is actuated in successive compression/decompression cycles to deliver liquid materials. The liquid does not penetrate through the diaphragm. The repeated compression/decompression changes the volume of a chamber in the pump head so that liquid enters through an inlet check valve during decompression and exits through an outlet check valve during compression, effectively pushing material through the chamber.   Two factors are presumed to have an influence on pump metering accuracy: the output check value spring size (Factor A), and  the pump speed (Factor B). These two factors are independent since each can be changed by a process operator without influencing the other factor, i.e. pump speed does not influence the check valve and vice-versa.

There are two pump speeds that can be used and two spring sizes that are reasonable.  Therefore there are four possible combinations as depicted in Table \ref{RoutDes}:

\begin{table}[]
	\begin{tabular}{rc|c|c|c|c|}
		\cline{3-6}
		\multicolumn{1}{c}{\textbf{}}                                                   & \multicolumn{1}{l|}{}                                        & \multicolumn{2}{c|}{\textbf{\begin{tabular}[c]{@{}c@{}}A\\ Pump  \\ Speed\end{tabular}}} & \multicolumn{2}{c|}{\textbf{\begin{tabular}[c]{@{}c@{}}B\\ Spring size\end{tabular}}} \\ \cline{3-6} 
		\multicolumn{1}{l}{}                                                            & \multicolumn{1}{l|}{}                                        & \textbf{-}                                      & \textbf{+}                                     & \textbf{-}                               & \textbf{+}                              \\ \hline
		\multicolumn{1}{|c|}{\begin{tabular}[c]{@{}c@{}}Std\\ Run\\ Order\end{tabular}} & \begin{tabular}[c]{@{}c@{}}Random\\ Run\\ Order\end{tabular} & 40 g/min                                          & 90 g/min                                         & 15 psi                                & 100 psi                                \\ \hline
		\multicolumn{1}{|r|}{\textbf{1}}                                                & \textbf{3}                                                   & \multicolumn{2}{c|}{40 g/min}                                                                      & \multicolumn{2}{c|}{15 psi}                                                     \\ \hline
		\multicolumn{1}{|r|}{\textbf{2}}                                                & \textbf{1}                                                   & \multicolumn{2}{c|}{40 g/min}                                                                      & \multicolumn{2}{c|}{100 psi}                                                      \\ \hline
		\multicolumn{1}{|r|}{\textbf{3}}                                                & \textbf{4}                                                   & \multicolumn{2}{c|}{90 g/min}                                                                      & \multicolumn{2}{c|}{15 psi}                                                     \\ \hline
		\multicolumn{1}{|r|}{\textbf{4}}                                                & \textbf{2}                                                   & \multicolumn{2}{c|}{90 g/min}                                                                      & \multicolumn{2}{c|}{100 psi}                                                      \\ \hline
	\end{tabular}\caption{Metering Pump Experiment Design}\label{RoutDes}
\end{table}








The experimenter has chosen to run four replicates of this experiment.  Four vessels are weighed at each run condition in the above table.  The data recorded is the mass of liquid delivered by the pump at the set speed and size of springs used. The goal of the experiment is to deliver a specific mass of material at the desired rate, and the response variable to be analyzed is the deviation in milligrams of the target mass delivered.. Table \ref{RouterData} summarizes the data collected.

\begin{table}[]
\begin{tabular}{lcc|r|r|r|r|r}
\cline{4-7}
                                                                                   & \multicolumn{1}{l}{}                                                                   & \multicolumn{1}{l|}{} & \multicolumn{4}{c|}{\textbf{Replicates}}                                                                                              & \multicolumn{1}{l}{}                                                                   \\ \hline
\multicolumn{1}{|c|}{\textbf{\begin{tabular}[c]{@{}c@{}}Run\\ Order\end{tabular}}} & \multicolumn{1}{c|}{\textbf{\begin{tabular}[c]{@{}c@{}}Cutting \\ Speed\end{tabular}}} & \textbf{Bit Size}     & \multicolumn{1}{c|}{\textbf{1}} & \multicolumn{1}{c|}{\textbf{2}} & \multicolumn{1}{c|}{\textbf{3}} & \multicolumn{1}{c|}{\textbf{4}} & \multicolumn{1}{c|}{\textbf{\begin{tabular}[c]{@{}c@{}}Mean \\ Response\end{tabular}}} \\ \hline
\multicolumn{1}{|l|}{1}                                                            & \multicolumn{1}{c|}{40 rpm}                                                            & 1/8th in              & 18.2                            & 18.9                            & 12.9                            & 14.4                            & \multicolumn{1}{r|}{16.1}                                                              \\ \hline
\multicolumn{1}{|l|}{2}                                                            & \multicolumn{1}{c|}{90 rpm}                                                            & 1/8th in              & 27.2                            & 24.0                            & 22.4                            & 22.5                            & \multicolumn{1}{r|}{24.0}                                                              \\ \hline
\multicolumn{1}{|l|}{3}                                                            & \multicolumn{1}{c|}{40 rpm}                                                            & 1/16th in             & 15.9                            & 14.5                            & 15.1                            & 14.2                            & \multicolumn{1}{r|}{14.9}                                                              \\ \hline
\multicolumn{1}{|l|}{4}                                                            & \multicolumn{1}{c|}{90 rpm}                                                            & 1/16th in             & 41.0                            & 43.9                            & 36.3                            & 39.9                            & \multicolumn{1}{r|}{40.3}                                                              \\ \hline
\end{tabular}\caption{Router Experiment Data Table}\label{RouterData}
\end{table}


\subsection{Analysis of the response data}
To analyze this data a simple implementation of Yates algorithm can be applied in an analysis matrix as shown in Table \ref{2fac2lev_matrix}.

\begin{center}
	% Please add the following required packages to your document preamble:
% \usepackage[table,xcdraw]{xcolor}
% If you use beamer only pass "xcolor=table" option, i.e. \documentclass[xcolor=table]{beamer}
\begin{table}[]
	\begin{tabular}{|c|l|c|l|l|l|l|l|l|l|l|l|}
		\hline
		\multicolumn{2}{|l|}{\begin{tabular}[c]{@{}l@{}}Random \\ Order\end{tabular}} & \multicolumn{2}{l|}{\begin{tabular}[c]{@{}l@{}}Standard\\ Order\end{tabular}} & \multicolumn{2}{l|}{\begin{tabular}[c]{@{}l@{}}Mean\\ Response\end{tabular}} & \multicolumn{2}{c|}{\textbf{A}}                                                   & \multicolumn{2}{c|}{\textbf{B}}                                   & \multicolumn{2}{c|}{\textbf{AB}}                                   \\ \hline
		\multicolumn{6}{|l|}{\cellcolor[HTML]{000000}}                                                                                                                                                                                               & \multicolumn{1}{c|}{\textbf{-}} & \multicolumn{1}{c|}{\textbf{+}}                 & \multicolumn{1}{c|}{\textbf{-}} & \multicolumn{1}{c|}{\textbf{+}} & \multicolumn{1}{c|}{\textbf{-}} & \multicolumn{1}{c|}{\textbf{+}} \\ \hline
		\multicolumn{2}{|c|}{\textbf{3}}                                              & \multicolumn{2}{c|}{\textbf{1}}                                               & \multicolumn{2}{l|}{}                                                        &                                 & \cellcolor[HTML]{000000}{\color[HTML]{000000} } &                                 & \cellcolor[HTML]{000000}        & \cellcolor[HTML]{000000}        &                                 \\ \hline
		\multicolumn{2}{|c|}{\textbf{1}}                                              & \multicolumn{2}{c|}{\textbf{2}}                                               & \multicolumn{2}{l|}{}                                                        &                                 & \cellcolor[HTML]{000000}{\color[HTML]{000000} } & \cellcolor[HTML]{000000}        &                                 &                                 & \cellcolor[HTML]{000000}        \\ \hline
		\multicolumn{2}{|c|}{\textbf{4}}                                              & \multicolumn{2}{c|}{\textbf{3}}                                               & \multicolumn{2}{l|}{}                                                        & \cellcolor[HTML]{000000}        &                                                 &                                 & \cellcolor[HTML]{000000}        &                                 & \cellcolor[HTML]{000000}        \\ \hline
		\multicolumn{2}{|c|}{\textbf{2}}                                              & \multicolumn{2}{c|}{\textbf{4}}                                               & \multicolumn{2}{l|}{}                                                        & \cellcolor[HTML]{000000}        &                                                 & \cellcolor[HTML]{000000}        &                                 & \cellcolor[HTML]{000000}        &                                 \\ \hline
		\multicolumn{4}{|r|}{\textbf{TOTAL:}}                                                                                                                         & \multicolumn{2}{l|}{}                                                        &                                 &                                                 &                                 &                                 &                                 &                                 \\ \hline
		\multicolumn{4}{|r|}{\textbf{N:}}                                                                                                                             & \multicolumn{2}{l|}{}                                                        &                                 &                                                 &                                 &                                 &                                 &                                 \\ \hline
		\multicolumn{4}{|r|}{\textbf{MEAN:}}                                                                                                                          & \multicolumn{2}{l|}{}                                                        &                                 &                                                 &                                 &                                 &                                 &                                 \\ \hline
		\multicolumn{4}{|r|}{\textbf{EFFECT:}}                                                                                                                        & \multicolumn{2}{l|}{\cellcolor[HTML]{000000}}                                & \multicolumn{2}{l|}{}                                                             & \multicolumn{2}{l|}{}                                             & \multicolumn{2}{l|}{}                                             \\ \hline
	\end{tabular}\caption{2-Factor 2-Level Analysis Matrix}\label{2fac2lev_matrix}
\end{table}
\end{center}

While the experimental runs are actually carried out in a randomized order, the analysis requires that the data be input as standard order as shown in Table \ref{2fac2lev_matrix}. For each run, enter the mean value of the response adjacent to each standard run number.  Note that the experimental data in Table \ref{RouterData} is listed in \textit{run order}. Run \#1 corresponds to standard order 2, run \#2 to standard order 4, etc. The sum of the mean response values is then calculated and entered in the cell just below the last mean response.  Below that is entered the total number of values N (in this example the total is 4). The final entry in this column is calculated by dividing the mean sum by the total number of values to obtain an overall average response.

In the subsequent columns under each main effect and interaction effect, simply copy the mean response for each run (row) across in the cells that are \textit{not shaded}.  Leave the shaded cells vacant in all cases. Again total a sum for each column and a total number of values, only counting cells that have values entered.  The total number of values should be half of what was entered in the first column.  Calculate mean responses for each column using the values in each respective column.

The final step is to calculate the net effect.  In each case, subtract the value of the high level (+) response from the low level (-) response.  The result may be either a positive or negative value depending upon the relative magnitude of the high and low level responses.  Enter the net effect in the  bottom row of the matrix.  The completed analysis matrix for \textit{The Router Experiment}, in Table \ref{Filled_2fac2lev}, provides an example of how to correctly complete this process.

% Please add the following required packages to your document preamble:
% \usepackage[table,xcdraw]{xcolor}
% If you use beamer only pass "xcolor=table" option, i.e. \documentclass[xcolor=table]{beamer}
\begin{sidewaystable}[]
	{\renewcommand{\arraystretch}{1.8} %<- modify value to suit your needs
\begin{tabular}{|c|l|c|l|c|l|r|r|r|r|r|r|}
\hline
\multicolumn{2}{|c|}{\begin{tabular}[c]{@{}c@{}}Random \\ \\ Run Order\end{tabular}} & \multicolumn{2}{c|}{\begin{tabular}[c]{@{}c@{}}Standard \\ \\ Order\end{tabular}} & \multicolumn{2}{c|}{\begin{tabular}[c]{@{}c@{}}Mean\\ Response\end{tabular}} & \multicolumn{2}{c|}{\textbf{A}}                                                   & \multicolumn{2}{c|}{\textbf{B}}                                   & \multicolumn{2}{c|}{\textbf{AB}}                                  \\ \hline
\multicolumn{6}{|l|}{\cellcolor[HTML]{000000}}                                                                                                                                                                                                          & \multicolumn{1}{c|}{\textbf{-}} & \multicolumn{1}{c|}{\textbf{+}}                 & \multicolumn{1}{c|}{\textbf{-}} & \multicolumn{1}{c|}{\textbf{+}} & \multicolumn{1}{c|}{\textbf{-}} & \multicolumn{1}{c|}{\textbf{+}} \\ \hline
\multicolumn{2}{|c|}{\textbf{3}}                                                     & \multicolumn{2}{c|}{\textbf{1}}                                                   & \multicolumn{2}{c|}{14.9}                                                    & 14.9                            & \cellcolor[HTML]{000000}{\color[HTML]{000000} } & 14.9                            & \cellcolor[HTML]{000000}        & \cellcolor[HTML]{000000}        & 14.9                            \\ \hline
\multicolumn{2}{|c|}{\textbf{1}}                                                     & \multicolumn{2}{c|}{\textbf{2}}                                                   & \multicolumn{2}{c|}{16.1}                                                    & 16.1                            & \cellcolor[HTML]{000000}{\color[HTML]{000000} } & \cellcolor[HTML]{000000}        & 16.1                            & 16.1                            & \cellcolor[HTML]{000000}        \\ \hline
\multicolumn{2}{|c|}{\textbf{4}}                                                     & \multicolumn{2}{c|}{\textbf{3}}                                                   & \multicolumn{2}{c|}{40.3}                                                    & \cellcolor[HTML]{000000}        & 40.3                                            & 40.3                            & \cellcolor[HTML]{000000}        & 40.3                            & \cellcolor[HTML]{000000}        \\ \hline
\multicolumn{2}{|c|}{\textbf{2}}                                                     & \multicolumn{2}{c|}{\textbf{4}}                                                   & \multicolumn{2}{c|}{24.0}                                                    & \cellcolor[HTML]{000000}        & 24.0                                            & \cellcolor[HTML]{000000}        & 24.0                            & \cellcolor[HTML]{000000}        & 24.0                            \\ \hline
\multicolumn{4}{|r|}{\textbf{TOTAL:}}                                                                                                                                    & \multicolumn{2}{c|}{95.3}                                                    & 31.0                            & 64.3                                            & 55.2                            & 40.1                            & 56.4                            & 38.9                            \\ \hline
\multicolumn{4}{|r|}{\textbf{N:}}                                                                                                                                        & \multicolumn{2}{c|}{4}                                                       & 2                               & 2                                               & 2                               & 2                               & 2                               & 2                               \\ \hline
\multicolumn{4}{|r|}{\textbf{MEAN:}}                                                                                                                                     & \multicolumn{2}{c|}{23.83}                                                   & 15.50                           & 32.15                                           & 27.60                           & 20.05                           & 28.20                           & 19.45                           \\ \hline
\multicolumn{4}{|r|}{\textbf{EFFECT:}}                                                                                                                                   & \multicolumn{2}{l|}{\cellcolor[HTML]{000000}}                                & \multicolumn{2}{c|}{\textbf{16.65}}                                               & \multicolumn{2}{c|}{\textbf{-7.55}}                               & \multicolumn{2}{c|}{\textbf{-8.75}}                               \\ \hline
\end{tabular} } \caption{Router Experiment Analysis Matrix}\label{Filled_2fac2lev}
\end{sidewaystable}


So what is mean by an ``effect'' of a factor (main or interaction)?  An effect is simply the change in the mean value of the response, $y$, when the factor of interest is changed from its low level to its high level.  The effects for each main and interaction factor are calculated by taking the difference of the average high and average low levels of a factor. By completing the table above, we accomplished this in an organized way. Table \ref{Filled_2fac2lev} reveals that the largest effect was A - Cutting Speed. While this ``manual'' method is very easy to carry out and provides a quantitative comparison of effects, we can get much more information out of this experimental work by analyzing the data using R statistics software (along with RStudio).

\subsection{Analysis using R}
The first task when doing an analysis in R is to properly input the experimental data in a form that is easy to analyze. Fortunately we can take advantage of spreadsheet software to enter the data in rows and columns. For the router experiment we enter the data as shown in Figure \ref{spread}.

\begin{figure}
\begin{center}
		\includegraphics[scale=1]{SpreadData}
\end{center}
	\caption{Experimental Data entered into a spreadsheet. Columns indicate runs and rows are replicated runs.}\label{spread}
\end{figure}

The row and column labels in the spreadsheet are there to help us with data entry, ensuring that we have the correct data representing each replicate of each run in the experiment that we designed. We also enter the data in the random run order as carried out experimentally (no need to sort by standard order). Now we develop the R code for the analysis:

\input{2Fac2Lev.R}

\noindent\textbf{Output}: (displayed in the console)

\noindent\texttt{ANOVA results:}

\begin{verbatim}
            Df Sum Sq Mean Sq F value   Pr(>F)    
A            1 1107.2  1107.2  23.742 0.000383 ***
B            1   41.9    41.9   0.899 0.361755    
A:B          1    1.1     1.1   0.023 0.883184    
Residuals   12  559.6    46.6                     
---
Signif. codes:  0 ‘***’ 0.001 ‘**’ 0.01 ‘*’ 0.05 ‘.’ 0.1 ‘ ’ 1
\end{verbatim}

The ANOVA results tell the same story as the evaluation by Yates algorithm in the manual method: Factor A (Cutting Speed) has the largest effect and is statistically significant. The ANOVA adds the statistical significance because the right-most column of the analysis, \textbf{Pr(>F)}, lists the probability that the group mean values are so similar that any effect is ``buried in the noise.'' This is the null hypothesis -- that they are the same. The default decision criterion is a probability at 5\%, or 0.05 as displayed in the table. Any value in the \textbf{Pr(>F)} column that is less than the critical value (p < 0.05) is considered a statistically significant effect. Given these guidelines it is easy to see that Factor A has a very low p value (and the *** denotes significance down to the o.1\% level, well below the 5\% criterion). Factor B and the A-B interaction effects are both greater than the default criterion, and thus are just noise.

Once the calculations are complete, it is helpful to visualize the results by depicting them graphically in two dimensions. First we generate an interaction plot to confirm what we learned in the ANOVA, shown in Figure \ref{intplot}. Next we look at two diagnostic plots: residuals versus fitted points, and a normal quantile-quantile plot shown together in Figure \ref{diagplot}. The residuals plot should appear as a random display of scattered points, as it helps determine whether there are violations of the assumptions of the statistical model. Here are some of the assumptions for a linear model such as is being used here:

\begin{enumerate}
	\item The estimate exists.
	\item The dependent variable approximates a multivariate normal distribution. 
	\item Unbiasedness: $E\hat{\beta} = \beta$.
	\item Consistency: $\hat{\beta}\rightarrow\beta$ as $n\rightarrow\infty$ ($n$ here is the size of a data sample).
	\item Efficiency: $Var(\hat{\beta})$ is smaller than $Var(\tilde{\beta})$ for alternative estimates $\tilde{\beta}$ of $\beta$.
	\item The ability to either approximate or calculate the distribution function of $\hat{\beta}$.
\end{enumerate}

However, what we observe in the left plot Figure \ref{diagplot} is \textit{heteroscedasticity}, meaning that there are subsets of observations that have different variability than others.\cite{Wikipedia2018} In other words, as the predicted outcome (by the linear model) gets larger, the variance grows. We would expect \textit{homoscedasticity}, where the variance level is stable for all predicted outcomes. This violates the consistency assumption mentioned above. While notable, it is quite common in the real world.


\begin{figure}
	\begin{center}
		\includegraphics[width=\linewidth]{IntPlot.eps}
	\end{center}
	\caption{Interaction plot of Cutting Speed versus Bit Size. There is no evidence of interaction since the lines do not intersect. The dashed line, representing the 1/16'' bit, exhibits higher vibration at both levels of cutting speed.}\label{intplot}
\end{figure}


\begin{figure}
	\begin{center}
		\includegraphics[width=\linewidth]{diags.eps}
	\end{center}
	\caption{Residuals plot (left), and the Normal Q-Q plot (right).}\label{diagplot}
\end{figure}

The right plot in Figure \ref{diagplot} tests the assumption of normally distributed values of the dependent variable (the observed responses). This is just a visual check; as opposed to a statistical evaluation of data assumed to be normally distributed. If our assumption is plausible then the points should fall approximately on a straight line. This is what we see in Figure \ref{diagplot}.

This experiment simultaneously evaluated two factors (cutting speed and bit size) at two levels each. A simple implementation of Yates algorithm revealed that cutting speed was the most significant factor and there was no evidence of interaction between cutting speed and bit size. A second and more sophisticated analytical approach used the R statistics software to analyze the experimental data. The ANOVA analysis yielded the same result and the modeling effort provided additional diagnostics to determine if the model assumptions were held. Either of these two analytical approaches is a vast improvement over testing factors one at a time.


\section{Replication}
In the example factorial design given in the previous section each possible combination was replicated \textit{four times}. Had only one observation been made for each run and excessive variability exists in the process being studied, the estimates for each response may be far from the true average values. By referring back to table \ref{tab3}, on page \pageref{tab3} the raw data for the router experiment clearly exhibit variability. The effects of variation are reduced by replicating the experiment.  This practice yields at least four benefits:

\begin{enumerate}
\item mean values are less variable than individual observations -- the distribution of means is always narrower than the distribution of the corresponding individual measurements,
\item without replication a single erroneous or outlier observation can distort the entire analysis,
\item replicated experiments provide sufficient data to estimate the amount of variability, and
\item data from replicated experiments can be used to evaluate which factors influence the mean response and which affect the variability (location and variation statistics).
\end{enumerate}

To properly replicate an experiment the full set of factor combinations must be repeated.  In the example given earlier in this chapter, the experimenter chose to replicate the experiment four times, and since there are four combinations of factors a total of 16 tests were conducted to collect all of the raw data.  Refer again to section \ref{router} to verify the number of individual observations that were collected.

\section{Three-factor design}
Three-factor factorial designs are two-level experiments denoted as $ 2^{3} $ designs. The number of experimental runs is determined by computing $ 2^{3} = 8 $ and the design matrix is given in Table \ref{tab4}.
\begin{table}[h]\caption{The $2^{3}$ factorial design}\label{tab4}
\begin{center}
\begin{tabular}{|l|c|c|c|}
\hline Run & A & B & C\\ 
\hline 1 & -- & -- & --\\ 
\hline 2 & -- & -- & \cellcolor{black!95}+\\ 
\hline 3 & -- & \cellcolor{black!95}+ & --\\ 
\hline 4 & -- & \cellcolor{black!95}+ & \cellcolor{black!95}+\\
\hline 5 & \cellcolor{black!95}+ & -- & --\\
\hline 6 & \cellcolor{black!95}+ & -- & \cellcolor{black!95}+\\
\hline 7 & \cellcolor{black!95}+ & \cellcolor{black!95}+ & --\\
\hline 8 & \cellcolor{black!95}+ & \cellcolor{black!95}+ & \cellcolor{black!95}+\\ 
\hline 
\end{tabular} 
\end{center}
\end{table}
All possible combination of factors at each level are included in the full factorial design, and as mentioned above  a three-factor experiment requires 8 trial conditions, or runs.  

\subsection{EXAMPLE: Metal Cutting Process}
A metal cutting process yields an unsatisfactory surface roughness that the process engineer desire to improve.  There are three factors that have been identified for this study: tool angle, depth of cut, and feed rate. Because there is concern to limit costs for this experiment only two replicates are budgeted for this study.  The design matrix with low and high levels for each factor is given in Table \ref{tab5}.

\begin{sidewaystable}\caption{Metal Cutting Experiment Design}\label{tab5}
\begin{center}
\begin{tabular}{|l|c|c|c|}
\hline Run & \textbf{tool angle} & \textbf{depth of cut} & \textbf{feed rate}\\ 
\hline 1 & $15^{o}$ & 0.025 in & 20 in/min \\ 
\hline 2 & $15^{o}$ & 0.025 in & 30 in/min \\ 
\hline 3 & $15^{o}$ & 0.040 in & 20 in/min \\ 
\hline 4 & $15^{o}$ & 0.040 in & 30 in/min \\
\hline 5 & $25^{o}$ & 0.025 in & 20 in/min \\
\hline 6 & $25^{o}$ & 0.025 in & 30 in/min \\
\hline 7 & $25^{o}$ & 0.040 in & 20 in/min \\
\hline 8 & $25^{o}$ & 0.040 in & 30 in/min \\ 
\hline 
\end{tabular} 
\end{center}
\end{sidewaystable}

The process engineer performed the 16 trials planned for this study and the results appear in Table \ref{tab6}.
\begin{sidewaystable}[h]\caption{Metal Cutting Experiment Data Table}\label{tab6}
\begin{center}
\begin{tabular}{|l|c|c|c|r|r|r|}
\hline Run & \textbf{tool angle} & \textbf{depth of cut} & \textbf{feed rate} & \multicolumn{2}{c|}{\textbf{Surface Roughness}} & \textbf{Mean}\\ 
\hline 1 & $15^{o}$ & 0.025 in & 20 in/min &  9 &  7 &  8.0\\ 
\hline 2 & $15^{o}$ & 0.025 in & 30 in/min & 10 & 12 & 11.0\\ 
\hline 3 & $15^{o}$ & 0.040 in & 20 in/min &  9 & 11 & 10.0\\ 
\hline 4 & $15^{o}$ & 0.040 in & 30 in/min & 12 & 15 & 13.5\\
\hline 5 & $25^{o}$ & 0.025 in & 20 in/min & 11 & 10 & 10.5\\
\hline 6 & $25^{o}$ & 0.025 in & 30 in/min & 10 & 13 & 11.5\\
\hline 7 & $25^{o}$ & 0.040 in & 20 in/min & 10 &  8 &  9.0\\
\hline 8 & $25^{o}$ & 0.040 in & 30 in/min & 16 & 14 & 15.0\\ 
\hline 

\end{tabular} 
\end{center}
\end{sidewaystable}
\subsection{Analysis of the response data}
The analysis matrix for a three-factor, two-level design is given in Figure \ref{fig5}  with the data filled in and calculations completed for the Metal Cutting Experiment.
\begin{figure}[h]\caption{Metal Cutting Experiment Analysis Matrix}\label{fig5}
\begin{center}
\includegraphics[scale=0.5]{metal}
\end{center}
\end{figure}
Inspection of the bottom row of the analysis matrix shows that factor C (feed rate) has the greatest effect upon surface roughness. The next largest effects are B (cut depth) and BC -- the latter two-factor interaction makes intuitive sense since the two parent main effects are large relative to the magnitude of the other effects.

\subsection{Three-factor design space}\label{DesignSpace}
Now that the main and interaction effects have been calculated it is helpful to visualize the responses in the geometry of the design space.  As we saw in section \ref{2facspace} a $2^{2}$ factorial design space was a (2-dimensional) square.  Accordingly, Figure \ref{fig6} shows the \textit{3-dimensional} $2^{3}$ factorial design space depicted as a cube.
\begin{figure}[h]\caption{Metal Cutting Experiment Design Space}\label{fig6}
\begin{center}
\includegraphics[scale=0.6]{metal-resp}
\end{center}
\end{figure}
\subsection{Interaction effects}
The $2^{3}$ factorial design has four possible interaction effects: AB, AC, BC, and ABC.  There are three binary interactions and one ternary.  As the number of factors in a factorial experiment grows, the number of effects that can be estimated also grows, as does the dimensionality of interactions.  In most cases where higher order interactions (three-factor interactions and higher) the \textbf{sparsity of effects principle} applies. This principle is that in most cases the system under study is dominated by main effects and lower-order (binary) interactions.  The three-factor and higher order interaction effects are extremely rare and are therefore negligible.\label{sparsity}

For the metal cutting process we will focus on the two-factor interactions AB, AC, and BC.  The corresponding interaction plots are given in Figure \ref{fig7}.
\begin{figure}[h]\caption{Metal Cutting Experiment Interaction Plots}\label{fig7}
\begin{center}
\includegraphics[scale=0.5]{metal-int}
\end{center}
\end{figure}
A quick comparison of the magnitudes of the AB and AC effects and the corresponding interaction plots above makes it evident that there is little significance to these interactions. In contrast, the BC interaction (depth of cut--feed rate) has an effect with a magnitude almost as large as the B main effect.  The interaction plot suggests that when feed rate is kept at the low level there is negligible change in surface roughness at either level of cut depth, but when feed rate is high the surface roughness significantly increases with deeper cuts.  This analysis makes it relatively easy for the process engineer to settle on the preferred setting for feed rate:  keep it \textit{low}.

\section{Four-factor design}\label{4fac}
The $2^{4}$ factorial design is perhaps the largest full-factorial design that is practical for most studies in an industrial setting.  The $ 2^{4} = 16 $ runs when replicated translate into a \textit{multiple} of 16 and this can sometimes be cost or time prohibitive.  See Table \ref{tab7} for the $2^{4}$ factorial design matrix.

\begin{table}[h]\caption{The $2^{4}$ factorial design}\label{tab7}
\begin{center}
\begin{tabular}{|l|c|c|c|c|}
\hline Run & A & B  & C  & D \\ 
\hline 1  & -- & -- & -- & --\\ 
\hline 2  & -- & -- & -- & + \\ 
\hline 3  & -- & -- & +  & --\\ 
\hline 4  & -- & -- & +  & + \\
\hline 5  & -- & +  & -- & --\\
\hline 6  & -- & +  & -- & + \\
\hline 7  & -- & +  & +  & --\\
\hline 8  & -- & +  & +  & + \\
\hline 9  & +  & -- & -- & --\\
\hline 10 & +  & -- & -- & + \\
\hline 11 & +  & -- & +  & --\\
\hline 12 & +  & -- & +  & + \\
\hline 13 & +  & +  & -- & --\\
\hline 14 & +  & +  & -- & + \\
\hline 15 & +  & +  & +  & --\\
\hline 16 & +  & +  & +  & + \\ 
\hline 
\end{tabular} 
\end{center}
\end{table}

As might be expected, there are many more interaction effects in this higher order design than in previous designs -- and all of them can be estimated. There are six binary, four ternary, and one quaternary for a total of eleven interactions.  As discussed in section \ref{sparsity}, anything higher than the two-factor (binary) interactions is extremely rare and can safely be ignored.

\subsection{EXAMPLE: Milling Operation}
An industrial milling operation is being evaluated for the quality of surface finish milling on a metal part.  The key parameters that could be controlled in this milling operation are shown in Table \ref{tab8}. Figure \ref{fig8} gives the filled analysis matrix for this experiment.

\subsection{Analysis of the response data}
Based on the magnitude of calculated effects, main effects A, B, and D stand out, and interaction effects AD and BD are significant contributors.  It is not surprising that the AD and BD interaction effects are significant given that the parent main effects are also strong. If we temporarily disregard the interaction effects and only consider main effects, the experiment indicates that a minimum (desirable) value for surface finish can be achieve with A set high (100 m/min), B set low (1 mm), and D set low (0.25 mm/tooth).\label{maineff}

\begin{table}[h]\caption{Milling Experiment Design}\label{tab8}
\begin{center}
\begin{tabular}{|l|c|c|c|c|}
\hline Run & Low Level (--) & High Level (+)  \\ 
\hline cutting speed & 80 m/min & 100 m/min  \\
\hline depth of cut & 1 mm & 8 mm \\
\hline cutter diameter & 100 mm & 200 mm\\
\hline feed/tooth & 0.25 mm/tooth & 0.65 mm/tooth \\
\hline
\end{tabular} 
\end{center}
\end{table}

\begin{sidewaysfigure}[h]\caption{Milling Experiment Analysis Matrix}\label{fig8}
\begin{center}
<<<<<<< HEAD
\includegraphics[width=\linewidth]{milling}
=======
\includegraphics[width=\textwidth]{milling}
>>>>>>> 1ff93b714085cfe057adea807a5a9f9aced6281f
\end{center}
\end{sidewaysfigure}

\subsection{Interaction effects}
The $2^{4}$ factorial design has a total of eleven interaction effects.  The sparsity of effects principle (See section \ref{sparsity}) applies to nearly half of these, leaving six two-factor interaction effects to consider.  In this example there are strong effects for factors AD and BD and so a look at the respective interaction plots, in Figure \ref{fig9}, can reveal additional insight into preferred settings.

Now considering the interactions in tandem with main effects suggests a different scenario.  The minimum response is obtained when both A and D are set low. Referring back to section \ref{maineff} when only main effects were considered it was presumed that A should be set \textit{high} and D set low.  Careful consideration of the results is needed here -- as is evident in Figure \ref{fig9} when A is high the average response is not as sensitive to changes in the level of D (the slope is less than when A is low).  Depending on the milling situation then, if factor D is hard to control (or if it were a noise factor) then it would be best to set A at its high level as a means to reduce process variability.

\begin{sidewaysfigure}[h]\caption{Milling Experiment Interaction Plots}\label{fig9}
\begin{center}

\includegraphics[width=\linewidth]{milling-int}

\end{center}
\end{sidewaysfigure}

The BD interaction plot in Figure \ref{fig9} is also an interesting situation. There is less sensitivity (lower slope) when B is set high, but the average response is much lower when B and D are both set to their low levels. The initial recommendation based on these results would be that A be set high, B and D set low with a suggestion that further study may be needed to investigate the interaction of A and B with D. The recommended settings were used in (standard order) trial runs 9 and 11 in Figure \ref{fig8}. The responses for those trial runs were 44 and 49, respectively.  The average of these two runs is 46.5 -- significantly less than the observed overall average response of 69.75 micro inches of surface roughness.